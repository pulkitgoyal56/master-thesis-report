% The Methods section should be brief but adequate to allow a qualified investigator to repeat the research. The animals, supplies and equipment used should be described in detail. All companies from which compute resources were obtained should be listed with their location. All methods of analysis and statistical testing must be identified and explained in detail. Studies employing animals have to identify that the experiments being reported were approved by the Institutional Animal Care and Use Committee.
\chapter{Methods}
\label{sec:methods}

We use a family of VLMs called CLIP \citep{clip} for semantic inference and a model-based planning controller using an improved Cross-Entropy Method (iCEM) \citep{icem} as our artificial agent.
This chapter describes the components of our approach, including the mathematical setup, the CLIP model, the intrinsic semantics rewards, and the iCEM controller.

\section{Preliminaries}
\label{sec:preliminaries}
% - Notation
We formulate the problem as a \emph{fully observable} Markov Decision Process (MDP) given by the tuple \((\cS, \cA, \cO, \theta, f, r, \bfs_0, \bfo_0)\), with state-space \(\cS \in \nR^{n_s}\), action-space \(\cA \in \nR^{n_a}\), observation-space (image-space) \(\cO \in \nR^{n_o}\), rendering function \(\theta: \cS \mapsto \cO\), transition kernel (environment model) \(f: \cS \times \cA \mapsto \cS\), reward function \(r: \cS \times \cO \mapsto \nR\), initial state \(\bfs_0 \in \cS\), and its corresponding initial image observation \(\bfo_0 = \theta(\bfs_0), \bfo_0 \in \cO\).

A trajectory \(\bmtau_t = \{(\bfs_t, \bfo_t, R_t, \bfa_t), (\bfs_{t+1}, \bfo_{t+1}, R_{t+1}, \bfa_{t+1}), \ldots \}\) at time \(t\) is a sequence of observation-action pairs starting from the current state \(\bfs_t\), where \(\bfs_k \in \cS\), \(\bfs_k \in \cO\), \(R_k \in \nR\), and \(\bfa_k \in \cA\).

In a typical RL setting, the returns of such a trajectory \(\bmtau_t\) are the discounted sum of rewards given by,
\begin{equation}
    \label{eq:rl-goal}
    g(\bmtau_t) = \sum_{k=0} \gamma^k R_{t+k},
\end{equation}
where \(R_t\) is the expected reward at time \(t\) and \(\gamma \in [0, 1]\) is the discount factor. The agent's goal is to learn an action policy that maximizes its expected reward;
\begin{equation}
    \label{eq:rl-policy}
    \pi^*_t = \argmax_{\pi} E_{\bmtau_t \sim \pi} \left[ g(\bmtau_t) \right],
\end{equation}
where the expectation is under trajectories \emph{rolled-out} with \(\bfa_t \sim \pi(\cdot \mid \bfs_t)\), \(\bfs_{t+1} = f(\bfs_t, \bfa_t)\), \(\bfo_t = \theta(\bfs_t)\), and \(R_t = r(\bfs_t, \bfo_t)\).

In our formulations, we consider a finite-horizon problem, where the agent is tasked with maximizing the expected return over a limited planning horizon \(h\).

We use boldface to denote 

\newpage
\section{Vision-Language Models}
\label{sec:vlms}

We define vision-language models (VLMs; \cite{vlmsurvey}) as functions capable of processing both language inputs \(\bml \in \cL\) and vision inputs \(\bmi \in \cI\), where \(\cL\) is a set of natural language strings/prompts and \(\cI \subseteq \cO\) is the space of 2D RGB images.

\subsection{CLIP}
\label{sec:clip}
CLIP (Contrastive Language Image Pretraining; \cite{clip}) is a family of vision-language models that have two main components -- an image encoder \(\psi_I: \cI \mapsto \cV\) and a language encoder \(\psi_L: \cL \mapsto \cV\), which generate unit magnitude vector embeddings for images and text, respectively, in the same high-dimensional latent space \(\cV \in \nR^{n_v}\), which can thus be compared to each other using a vector similarity metric \(\zeta: \cI \times \cL \mapsto \nR\), e.g. cosine similarity (\(S_c\), essentially dot product).

For a label \(\bfl \in \cL\) and an image \(\bfi \in \cI\), the similarity is given by,
\begin{equation}
    \label{eq:clip-similarity}
    \zeta(\bfi, \bfl) = S_c(\psi_I(\bfi), \psi_L(\bfl)) = \dfrac{\psi_I(\bfi) \cdot \psi_L(\bfl)}{\| \psi_I(\bfi) \| \| \psi_L(\bfl) \|} = \psi_I(\bfi) \cdot \psi_L(\bfl).
\end{equation}

Eponymously, the CLIP image and text encoders are jointly trained using a contrastive loss which directs the model to semantically align the correct image and text pairs (in the latent space) by increasing their similarity while simultaneously pushing apart the embeddings of the mismatched pairs by decreasing their similarity.

Given a batch of \(N\) image-caption pairs; images \(\bmi^{(N)} = \{ \bfi_1, \dots, \bfi_N \}\) and captions \(\bml^{(N)} = \{ \bfl_1, \dots, \bfl_N \}\), such that the corresponding entries are paired, CLIP first computes the matrix of cosine similarities, \(\zeta^{(N)}(\bmi, \bml) \in \nR^{N \times N}\), where each item \(\zeta^{(N)}_{j,k}\) computes the cosine similarity between the image \(\bfi_j\) and caption \(\bfl_k\).
The training batch loss is then computed as the difference between the similarities corresponding to the correct image-caption pairs and the mismatched caption-image pairs, which is then minimized.

There are several variants of CLIP, with different sizes and different architectures for its image encoder (vision transformer \emph{ViT}; \cite{vit}, or residual network \emph{ResNet}; \cite{resnet}).
We used the vision transformer variant (\texttt{ViT-L/14}) for most of our experiments as it gave a good balance of performance and computational efficiency.
It takes image inputs of size \(224 \times 224\) and tokenizes them in patch sizes of \(14 \times 14\) for the vision transformer.
The comparison of the different models from the original paper for our use case is presented in the appendix \secref{sec:clip-comparison}.

\subsubsection{CLIP as Zero-Shot Image Classifier}
\label{sec:clip-classifier}
CLIP can be used as a zero-shot classifier by comparing the embeddings of a given image and a set of prompts to predict the likelihood of the image belonging to all of the classes.

Given an image \(\bfi\) and a set of text prompts \(\bml\), the likelihood of the image belonging to the class described by a prompt \(\bfl_k \in \bml\) is given by the softmax function over the similarity scores of the image and the prompt embeddings,
\begin{equation}
    \label{eq:clip-dist}
    p(\bfl_k; \bfi, \bml, T) = \dfrac{\exp(\zeta(\bfi, \bfl_k)/T)}{\sum_{\bfl_j \in \bml} \exp(\zeta(\bfi, \bfl_j)/T)},
\end{equation}
where \(T \in \nR^+\) is the temperature hyperparameter that controls the flatness of the distribution.
This is an important parameter as it directly controls the quality of our entropy reward.
We usually use the temperature \(0.01\) as suggested in the original paper for our experiments, which we also found to work much better than other values.
See \secref{sec:reg-temperature} for details.

\subsubsection{Prompts for CLIP}
\label{sec:prompt-engineering}
Ideally, CLIP should abstract away frivolous phrasing to focus on the semantics of a prompt, but in practice, due to inherent ambiguity in natural language, the exact phrasing can still influence its inference \citep{waffleclip}. 
We experimented with this in our investigations.

For objective analysis, we break the structure of a language prompt \(\bfc \in \cL\) representing a creative possibility \(c\) in an environment as ``\(\{\text{prefix}\}\{c\}\{\text{suffix}\}\)'', where ``prefix'' and ``suffix'' are optional context phrases. For example, if \(c =\) ``apple'', then the prompt could be ``picture of an apple, on a white background'', where ``picture of an '' is the prefix and ``, on a white background'' is the suffix.

\section{Semantic Rewards for Free Play}
\label{sec:semantics-reward}
This section describes the mathematical formulation of the intrinsic semantic reward we developed to encourage the agent to explore its environment in a semantically expressive manner, akin to free play in humans.
It also gives the necessary background on the additional intrinsic regularity reward that complements it.

\subsection{Semantics Entropy Reward}
\label{sec:entropy-reward}
% - Entropy reward and its regularization hyperparameters derived from the ZEST formulation

Mathematically, in its simplest form, the semantics reward \(r_{\text{semantics-entropy}}: \cO \mapsto \nR\) is formulated as the negative entropy of the likelihood of the predictions of the VLM for an image observation \(\bfo \in \cO\) over a set of \(n_c\) creative possibilities \(\bmc = \{ \bfc_1, \bfc_2, \ldots, \bfc_{n_c} \}\);
We call this the \emph{semantics-entropy reward},
\begin{equation}
    \label{eq:entropy-reward}
    r_{\text{semantics-entropy}}(\bfo; \bmc, T) := -\cH(p(\bmc \mid \bfo; T)) = \sum_{\bfc_k \in \bmc} p(\bfc_k \mid \bfo; \bmc, T) \log p(\bfc_k \mid \bfo; \bmc, T).
\end{equation}
% 
\subsubsection{Goal-Baseline Regularization}
\label{sec:goal-baseline}
To improve the reward quality, we further extend this formulation by adopting a combination of suggestions from other studies on VLMs as a source of goal-conditioned rewards.

The most trivial way to use CLIP as a goal-conditioned reward is to directly use the similarity metric from eq. \eqref{eq:clip-similarity} as the reward function.
Given a goal description \(\bfg \in \cL\), the reward for an image observation \(\bfo \in \cO\) is given by,
\begin{equation}
    \label{eq:trivial-reward}
    r_{\text{clip}}(\bfo; \bfg) = S_c(\psi_I(\bfo), \psi_L(\bfg)).
\end{equation}
This captures the projection of the image observation embedding \(\bfo\) onto the direction spanned by the goal embedding \(\bfg\).

\cite{zest} developed an alternative method called \emph{ZeST} (zero-shot task specification) which employs \emph{delta embeddings} that use the difference in embeddings between the desired and initial/\emph{baseline} configurations, for both image and text inputs, to directionally remove irrelevant parts of the CLIP representation.
The goal-baseline regularized reward function in this case is the similarity between the delta embeddings,
\begin{equation}
    \label{eq:zest}
    r_{\text{zest}}(\bfo; \bfo_b, \bfg, \bfg_b) = S_c((\psi_I(\bfo) - \psi_I(\bfo_b)), (\psi_L(\bfg) - \psi_L(\bfg_b))).
\end{equation}
where \(\bfg_b \in \cL\) is a natural language description of the initial/baseline environment image observation \(\bfo_b = \bfo_0\).

Intuitively, proving a baseline sets the context of the setup, and the direction from a baseline \(\bfg_b\) to goal \(\bfg\) captures the desired change from the environment's image baseline \(\bfo_b\) to the target state.
As an example, consider the case where the goal is to draw a circle starting from a blank canvas i.e. the goal \(\bfg\) is something like ``a circle on a white background''.
Then the baseline \(\bfg_b\) could be ``a blank canvas'' and the image observation baseline \(\bfo_b\) would be the starting blank canvas image.

\cite{negprompt} further proposed a negative prompt formulation that used the negations of goal states, instead of descriptions of the initial states, as goal baselines.
In the example, this would be ``not a circle on a white background''.

These delta embeddings considerably improved performance in language-guided goal-conditioned tasks by providing a more focused reward signal.
However, it is not certain that the direction from the baseline to the goal captures all the relevant information.
Therefore, instead of using one or the other, \cite{vlmrm} further made a trade-off between the two projections in their \emph{VLM-RM} method and proposed a goal-baseline regularization technique that considered both -- the similarity of the image observation to the target and also the direction from the baseline towards the goal in the CLIP embedding space.

\emph{VLM-RM} does not use delta embeddings or baselines for the image input, but only for the text input.
Their formulation is given by,
\begin{equation}
    \label{eq:vlmrm}
    r_{\text{goal-reg}}(\bfo; \bfg, \bfg_b, \gamma) = 1 - \dfrac{1}{2}\ \norm{\gamma\ \text{proj}_{V}\psi_I(\bfo) + (1 - \gamma)\ \psi_I(\bfo) - \psi_L(\bfg)}^2_2,
\end{equation}
where \(V := \frac{\psi_I(\bfg) - \psi_L(\bfg_b)}{\norm{\psi_I(\bfg) - \psi_L(\bfg_b)}}\) is the direction spanned by the vector between the embeddings of \(\bfg\) and \(\bfg_b\), \(\text{proj}_{V}\psi_I(\bfo) := (\bfo \cdot V)\,V\) is the projection of the image observation embedding \(\bfo\) onto this direction, and \(\gamma \in [0, 1]\) is a hyperparameter to control the trade-off/regularization strength.
Note that the magnitude of the projection \(\norm{\text{proj}_{V}\psi_I(\bfo)}_2\) is equal to the cosine similarity \(\cos(\vartheta) := S_c(\psi_I(\bfo), (\psi_L(\bfg) - \psi_L(\bfg_b)))\), and \(\sin(\vartheta)\) is the projection of \(\psi_I(\bfo)\) perpendicular to this direction.
The authors found this method to be relatively robust to changes in \(\gamma\) with most intermediate values being better than \(0\) or \(1\).
% 
% \begingroup
% \allowdisplaybreaks
\vspace{-1.5pt}
\begin{align}
    % r_{\text{goal-reg}}(\bfo; \bfg, \bfg_b, \gamma) &= 1 - \dfrac{1}{2}\ \norm{\gamma\ \text{proj}_{V}\psi_I(\bfo) + (1 - \gamma)\ \psi_I(\bfo) - \psi_L(\bfg)}^2_2, \label{eq:vlmrm}
    % \intertext{where \(V\) is the direction spanned by \(\psi_I(\bfg_b) - \psi_L(\bfg)\), \(\text{proj}_{V}\psi_I(\bfo)\) is the projection of the image observation embedding \(\bfo\) onto this direction, and \(\gamma \in [0, 1]\) is a hyperparameter to control the regularization/trade-off strength.
    % Note that the magnitude of the projection \(\norm{\text{proj}_{V}\psi_I(\bfo)}_2\) is equivalent to the cosine similarity \(\cos(\vartheta) := S_c(\psi_I(\bfo), (\psi_L(\bfg) - \psi_L(\bfg_b)))\), and \(\sin(\vartheta)\) is the projection of \(\bfo\) perpendicular to this direction.
    % The authors found this method to be relatively robust to changes in \(\gamma\) with most intermediate values being better than \(0\) or \(1\).}
    \shortintertext{For \(\gamma = 0\), the original CLIP similarity metric from eq. \eqref{eq:clip-similarity} is recovered as an unregularized goal-conditioned reward function,}
    r_{\text{goal-reg}}(\bfo; \bfg, \bfg_b, 0) &= 1 - \dfrac{1}{2}\ \norm{\psi_I(\bfo) - \psi_L(\bfg)}^2_2 = S_c(\psi_I(\bfo), \psi_L(\bfg)), \label{eq:vlmrm-0}
    \intertext{since \(\norm{\psi_I(\bfg_b)}_2 = \norm{\psi_L(\bfg)}_2 = 1\).
    For \(\gamma = 1\), there is a trade-off of components of \(\bfo\) orthogonal and parallel to ``\(\bfg - \bfg_b\)''.
    % Thus, it only retains the components that align with the direction from the baseline to the goal.
    % \texttt{-- A claim from the original paper; is incorrect!}
    }
    r_{\text{goal-reg}}(\bfo; \bfg, \bfg_b, 1) &= 1 - \dfrac{1}{2}\ \norm{\text{proj}_{V}\psi_I(\bfo) - \psi_L(\bfg)}^2_2\\
    &= 1 - \dfrac{\norm{\text{proj}_{V}\psi_I(\bfo)}^2_2 + 1 - 2\ \text{proj}_{V}\psi_I(\bfo) \cdot \psi_L(\bfg)}{2}\\
    &= \underbrace{\dfrac{1 - \norm{\text{proj}_{V}\psi_I(\bfo)}^2_2}{2}}_{\propto\ \sin^2(\vartheta)} + \underbrace{\dfrac{\psi_I(\bfo) \cdot (\psi_L(\bfg) - \psi_L(\bfg_b))}{2}}_{\propto\ \cos(\vartheta)}. \label{eq:vlmrm-1}
\end{align}
% (Refer to eq. \eqref{eq:vlmrm-equivalence-start} to \eqref{eq:vlmrm-equivalence-end} for more details on the derivation.)\\
% Mathematically, this is not equivalent to eq. \eqref{eq:zest} (with no image baseline), because of the additional term, but it is similar in spirit.
% \endgroup

\subsubsection{Regularized Semantics Entropy Reward}
\label{sec:entropy-reward-reg}
We combine these ideas to derive a \emph{baseline regularized semantics-entropy reward}.
Let \(\bmp = \text{proj}_{V}\psi_I(\bfo)\), \(\bmo = \psi_I(\bfo),\ \bmg = \psi_L(\bfg),\ \bmg_b = \psi_L(\bfg_b)\) for brevity, then using the property \(\norm{\bmo}_2 = \norm{\bmg}_2 = \norm{\bmg_b}_2 = 1\) and the results \(\bmp \cdot \bmo = \norm{p}^2_2\) and \(\bmp \cdot \bmg = (\bmg - \bmg_b) \cdot \bmo\ /\ 2\), the VLM-RM reward can be simplified as,
\vspace{-1.5pt}
\begin{align}
    &1 - \dfrac{1}{2} \norm{\gamma\ \bmp + \left(1 - \gamma\right) \bmo - \bmg}^2_2 \tag{\(r_{\text{goal-reg}}\), from \ref{eq:vlmrm}}\\
    \label{eq:vlmrm-equivalence-start}
    &= 1 - \dfrac{1}{2} \Big[\norm{\gamma\ \bmp + \left(1 - \gamma\right) \bmo}^2_2 + \cancelto{1}{\norm{\bmg}^2_2} &&- 2 \left(\gamma\ \bmp + \left(1 - \gamma\right) \bmo\right) \cdot \bmg\Big]\\
    &= \dfrac{1 - \norm{\gamma\ \bmp + \left(1 - \gamma\right) \bmo}^2_2}{2} &&+ \gamma\ \bmp \cdot \bmg + \left(1 - \gamma\right) \bmo \cdot \bmg\\
    &= \dfrac{1 - \norm{\gamma \left(\bmp - \bmo\right) + \bmo}^2_2}{2} &&+ \dfrac{\gamma}{2} \left(\bmg - \bmg_b\right) \cdot \bmo + \left(1 - \gamma\right) \bmg \cdot \bmo\\
    &= \dfrac{1 - \gamma^2\ \norm{\bmp - \bmo}^2_2 - 2 \gamma \left(\bmp - \bmo\right) \cdot \bmo - \cancelto{1}{\norm{\bmo}^2_2}}{2} &&+ \left(\dfrac{\gamma}{2} \left(\bmg - \bmg_b\right) + \left(1 - \gamma\right) \bmg\right) \cdot \bmo\\
    &= \dfrac{-\gamma^2 \left(\norm{\bmp}^2_2 + \norm{\bmo}^2_2 - 2\ \bmp \cdot \bmo\right) - 2 \gamma \left(\bmp \cdot \bmo - \norm{\bmo}^2_2\right)}{2} &&+ \left(\dfrac{\gamma}{2}\ \bmg - \dfrac{\gamma}{2}\ \bmg_b + \bmg - \gamma\ \bmg\right) \cdot \bmo\\
    % &= \dfrac{-\gamma^2 \big(\norm{\bmp}^2_2 + \cancelto{1}{\norm{\bmo}^2_2} - 2\ \bmp \cdot \bmo\big) - 2 \gamma \big(\bmp \cdot \bmo - \cancelto{1}{\norm{\bmo}^2_2}\big)}{2} &&+ \left(\dfrac{\gamma}{2}\ \bmg - \dfrac{\gamma}{2}\ \bmg_b + \bmg - \gamma\ \bmg\right) \cdot \bmo\\ % above with cancellations
    &= \dfrac{-\gamma^2 \left(\norm{\bmp}^2_2 + 1 - 2\ \norm{\bmp}^2_2\right) - 2 \gamma \left(\norm{\bmp}^2_2 - 1\right)}{2} &&+ \left(\bmg - \dfrac{\gamma}{2}\ \bmg - \dfrac{\gamma}{2}\ \bmg_b\right) \cdot \bmo\\
    &= \dfrac{-\gamma^2 + 2\gamma}{2} + \dfrac{\gamma^2\ \norm{\bmp}^2_2 - 2 \gamma\ \norm{\bmp}^2_2}{2} &&+ \left(\left(1 - \dfrac{\gamma}{2}\right) \bmg - \dfrac{\gamma}{2}\ \bmg_b\right) \cdot \bmo\\ % additional step
    %% Option 1 -- p^2 term
    &= \gamma\left(1 - \dfrac{\gamma}{2}\right)\left(1 -\norm{\bmp}^2_2\right) &&+ \left(\left(1 - \dfrac{\gamma}{2}\right) \bmg - \dfrac{\gamma}{2}\ \bmg_b\right) \cdot \bmo\\
    &= \underbrace{\gamma\left(1 - \dfrac{\gamma}{2}\right)\left(\sin\left(\vartheta\right)\right)}_{\text{Term \#1 (\(\perp\))}} &&+ \underbrace{\left(\left(1 - \dfrac{\gamma}{2}\right) \bmg - \dfrac{\gamma}{2}\ \bmg_b\right) \cdot \bmo}_{\text{Term \#2 (\(\parallel\))}}.
    %% Option 2 -- constant term separated
    % &= \gamma\left(1 - \dfrac{\gamma}{2}\right) -\ \gamma \left(1 - \dfrac{\gamma}{2}\right)\norm{\bmp}^2_2 &&+ \left(\left(1 - \dfrac{\gamma}{2}\right) \bmg - \dfrac{\gamma}{2}\ \bmg_b\right) \cdot \bmo\\
    % &= \underbrace{\gamma\left(1 - \dfrac{\gamma}{2}\right)}_{\textstyle\text{constant}} -\ \underbrace{\gamma \left(1 - \dfrac{\gamma}{2}\right)\left(\bmo \cdot \dfrac{\left(\bmg - \bmg_b\right)}{\norm{\bmg - \bmg_b}_2}\right)^2}_{\text{Term \#1 (\(\perp\))}} &&+ \underbrace{\left(\left(1 - \dfrac{\gamma}{2}\right) \bmg - \dfrac{\gamma}{2}\ \bmg_b\right) \cdot \bmo}_{\text{Term \#2 (\(\parallel\))}}.
    \label{eq:vlmrm-equivalence-end}
\end{align}
Term \#1 captures the projection perpendicular to the direction from the baseline to the goal and Term \#2 captures the trade-off of projection along this direction and the goal.
To simplify, we consider only the latter in our formulation as it contains the desired information about the goal.
We adopt this term as the basis to add baseline regularization to the semantics-entropy reward and consider similar delta embeddings for the image observation space as well.

First, let us formulate a regularized similarity metric for an image observation \(\bfi \in \cI\) with baseline image \(\bfi_b \in \cI\) and target label \(\bfl \in \cL\) with baseline \(\bfl_b \in \cL\);
\begin{equation}
    \label{eq:clip-similarity-reg}
    \begin{split}
        \hat{\zeta}(\bfi, \bfl; \bfi_b, \bfl_b, \alpha, \beta)
        &= S_c(
            ((1 - \beta)\ \psi_I(\bfi) - \beta\ \psi_I(\bfi_b)),
            ((1 - \alpha)\ \psi_I(\bfl) - \alpha\ \psi_I(\bfl_b))
        )
    \end{split}
\end{equation}
where \(\alpha, \beta \in [0, 0.5]\) are the regularization hyperparameters homologous to \(\gamma / 2\) in eq. \eqref{eq:vlmrm-equivalence-end}, for text and image inputs respectively, and \(T \in \nR^+\) is the temperature hyperparameter.

The likelihood distribution \(\hat{p}\) of the predictions of the VLM for the image observation \(\bfo\) over the creative possibilities \(\bmc\). For \(\bfc_k \in \bmc\) is then given by,
\begin{equation}
    \label{eq:clip-dist-reg}
    \hat{p}(\bfc_k \mid \bfo; \bfo_b, \bmc, \bfc_b, \alpha, \beta, T) = \dfrac{\exp(\hat{\zeta}(\bfo, \bfc_k; \bfo_b, \bfc_b, \alpha, \beta)/T)}{\sum_{\bfc_j \in \bmc} \exp(\hat{\zeta}(\bfo, \bfc_k; \bfo_b, \bfc_b, \alpha, \beta)/T)},
\end{equation}
The regularized semantics-entropy reward is thus defined as,
\begin{align}
    r_{\text{semantics-entropy-reg}}(\bfo; \bfo_b, \bmc, \bfc_b, \alpha, \beta, T)
    &:= -\cH(\hat{p}(\bmc \mid \bfo; \bfo_b, \bfc_b, \alpha, \beta, T)) \label{eq:entropy-reward-reg}\\
    &= \sum_{\bfc_k \in \bmc} \hat{p}(\bfc_k \mid \bfo; \bfo_b, \bmc, \bfc_b, \alpha, \beta, T) \log \hat{p}(\bfc_k \mid \bfo; \bfo_b, \bmc, \bfc_b, \alpha, \beta, T). \nonumber
\end{align}
The text baseline could either be the initial state description or the negation of the goal.

\newpage
\subsection{Regularity Reward}
\label{sec:regularity-reward}
In addition to the entropy reward, the semantic reward also consists of regularity as an intrinsic reward \citep{rair}, which encourages the agent to create regular and symmetric patterns in its creations, which could promote semantic expression.
The authors described the regularity reward \emph{RaIR} as ``(the) redundancy in the description of the situation, to measure the degree of sub-structure recurrence''.

Mathematically, it is considered that the state space can be factorized into a composition of the different entities in the environment and the configuration of the agent, \(\cS = (\cS_{\text{obj}})^N \times \cS_{\text{agent}}\), where \(\cS_{\text{obj}} \in \nR^d\), such that \(n_s = N \cdot d\).
This allows the\emph {RaIR} reward \(r_{\text{RaIR}}: \cS \mapsto \nR\) to be formulated in terms of multiplicities \(m: \cX \mapsto \nN\) (counts of occurrence) of the sub-structures \(\cX\) in the arrangement of the different entities of the environment in the state \(\bfs\),
\begin{equation}
    \label{eq:rair}
    r_{\text{RaIR}}(\bfs; h) := -\cH(\phi(\bfs; h)) = \sum_{\cX \in \phi(\bfs)} p(\cX; h) \log p(\cX; h),
\end{equation}
where \(\phi\) is a factorization function that decomposes a state \(\bfs\) into a multiset of its unique sub-structures given the hyperparameters \(h\) (resolution, bidirectionality),
\begin{equation}
    \label{eq:rair-factorization}
    \phi : (\cS_{obj})^{N} \mapsto \{ \cX_1^{m(\cX_1)}, \cX_2^{m(\cX_2)}, \ldots \},
\end{equation}
and \(p(\cX; h)\) is the probability of a sub-structure \(\cX \in \phi(\bfs; h)\) in the state \(\bfs\) calculated as,
\begin{equation}
    \label{eq:rair-probability}
    p(\cX; h) = \dfrac{m(\cX)}{\sum_{\cX' \in \phi(\bfs; h)} m(\cX')}.
\end{equation}
% 
For our purposes, following insights from \cite{symmetry} and \cite{compositional}, we use the relational formulation of \emph{RaIR} which captures the pairwise relationships between the entities of the environment in the state \(\bfs\) by calculating the distances between them in the discretized coordinate space;
\begin{equation}
    \label{eq:rair-relational}
    \phi(\bfs) = \{ \cX^{m(\cX)} : \cX = \lfloor s^{(j)} - s^{(k)} \rceil \mid s^{(j)}, s^{(k)} \in \bfs \},
\end{equation}
where the \(\lfloor \cdot \rceil\) operator groups the distances into discrete bins based on the configured resolution.
For more details on the parameters and formulation, please refer to \secref{sec:regularity-reward-details}.

\subsection{Complete Intrinsic Semantics Reward}
\label{sec:complete-semantics-reward}
The complete intrinsic semantics reward \(r_{\text{semantics}}\) is then given as the weighted sum of the regularity reward \(r_{\text{RaIR}}\) and the regularized semantics-entropy reward \(r_{\text{semantics-entropy-reg}}\);
\begin{equation}
    \label{eq:semantics-reward}
    r_{\text{semantics}}(\bfs, \bfo; \bfo_b, \bmc, \bfc_b, \alpha, \beta, T) =r_{\text{semantics-entropy-reg}}(\bfo; \bfo_b, \bmc, \bfc_b, \alpha, \beta, T) + \lambda\ r_{\text{RaIR}}(\bfs),
\end{equation}
where \(\lambda\) is a hyperparameter that controls the trade-off between the two rewards.

To be able to properly compare the two rewards, we normalize the regularity reward by the maximum possible entropy of the multiset \(\log\left(\binom{N}{2}\right)\), which is equal to the entropy of a uniform distribution over the \(\binom{N}{2}\) possible pairwise distances in the state space.
This constrains it to the range \([0, 1]\).

Similarly, the semantics-entropy reward is normalized by the maximum possible entropy of the likelihood distribution over the creative possibilities, which is \(\log(n_c)\), where \(n_c\) is the number of creative possibilities.

\section{iCEM}
\label{sec:icem}
% The Cross-Entropy Method (CEM) is an adaptive importance sampling procedure.

In the Model-Based Reinforcement Learning (MBRL) paradigm, the agent learns a model \(f\) of the MDP dynamics from interactions with the environment and uses it to predict the consequences of its actions which enables it to plan for the best policies.
Such policies can be learned with universal function approximators such as neural networks using RL algorithms or planning methods.

The improved Cross-Entropy Method (iCEM) is one such planning method.
It is a zeroth-order Monte-Carlo trajectory optimizer that works in a Model Predictive Control (MPC) loop to solve finite-horizon planning problems, i.e. it iteratively re-plans after every step in the environment by (1) coming up with several action sequences (by sampling a multi-variate distribution) to \emph{think} a few steps ahead, (2) imagining/predicting the future states resulting from these actions (using the model of the environment) and assessing their expected returns (using the reward function), (3) picking out the best action sequences, (4) improving the \emph{thinking} method based on these elite sequences, and (5) taking the first step of the best action sequence.

Precisely, given a finite planning horizon \(h\), iCEM uses a clipped colored-noise Gaussian distribution of dimension \(n_a \times h\) over action sequences (\(\bma^{(h)} = \{ \bfa_0, \bfa_1, \ldots, \bfa_{h-1} \}\)), given by,
\begin{equation}
    \label{eq:icem-distribution}
    p(\bma^{(h)}) = \text{clip}(\bmmu_t + \cC^\beta(n_a, h) \odot \bmsigma_t^2),
\end{equation}
where \(\bmmu_{t, \bmsigma_{t}} \in \nR^{n_a \times h}\) are the mean and standard deviation of the sampling distribution at time \(t\) respectively, and \(\cC^\beta(n_a, h)\) is a sampling function that returns \(n_a\) (one for each action dimension) sequences of length \(h\) sampled from colored noise normal distribution with exponent \(\beta\) (\(= 1\) in our experiments; pink noise), zero mean, and unit variance.

At every timestep \(t\), there are multiple \emph{inner} iterations of the following calculations.
At inner iteration \(i\),
(1) starting with a sampling distribution \( p(\bmmu_{t}^{(i - 1)},\bmsigma_{t}^{(i - 1)} )\), a fixed-length (\(n\)) set of action sequences (\(\cA_t = \{ \bma^{(h)}_1, \bma^{(h)}_2, \ldots, \bma^{(h)}_{n} \}\)) is sampled from this distribution.
These samples are augmented with a fraction of the best action sequences (called the \emph{elite-set} \(\cE_t^{(i - 1)}\)) from the previous inner iteration (called \emph{keep-elites}) or the previous timestep (called \emph{shift-elites}) if \(i = 0\). 
Additionally, to ensure that the mean action of the distribution is accessible to the controller, in the last inner iteration \(m\), the mean of the distribution \(\bmmu_t\) is added to the elite-set \(\cA^+\);
\begin{equation}
    \label{eq:icem-augment}
    \cA^+_t = \cA_t \cup \begin{cases}
            \cE_{t - 1}^{(m)} & \text{if } i = 0,\\
            \cE_t^{(m - 1)} \cup \{ \bmmu_t^{(m)} \} & \text{if } i = m,\\
            \cE_t^{(i - 1)} & \text{otherwise}.
    \end{cases}
\end{equation}

(2) Every action sequence in this augmented set \(\cA^+_t\) is then rolled out (\emph{in imagination}) using the environment model \(f\) and rendering function \(\theta\) from the current state \(\bfs_t\) to predict their corresponding expected states and image observations respectively,
\vspace{-1.5pt}
\begin{align}
    \cS_t &= \{ \{ \bfs_{t+1}, \bfs_{t+2}, \ldots, \bfs_{t+h} \} \}_{j=1}^{n}, \label{eq:icem-rollout}\\
    \cO_t &= \{ \{ \bfo_{t+1}, \bfo_{t+2}, \ldots, \bfo_{t+h} \} \}_{j=1}^{n}, \label{eq:icem-observations}
    \intertext{where \(\bfs_t = f(\bfs_{t-1}, \bfa_{t-1})\) and \(\bfo_t = \theta(\bfs_t)\). The predicted states are evaluated according to the reward function \(r\) to get their corresponding rewards,}
    \cR_t &= \{ \{ R_{t+1}, R_{t+2}, \ldots, R_{t+h} \} \}_{j=1}^{n}, \label{eq:icem-rewards}
\end{align}
where \(R_{t} = r(\bfs_t, \bfo_t) = \sum_k \lambda_k r_k(\bfs_t, \bfo_t)\) is the weighted sum of all the rewards.
The corresponding trajectories are,
\begin{equation}
    \label{eq:icem-trajectories}
    \cT^{(n \times h)}_t = \{\{(\bfs_t, \bfo_t, R_t, \bfa_t), \ldots, (\bfs_{t+h-1}, \bfo_{t+h-1}, R_{t+h-1}, \bfa_{t+h-1}), (\bfs_{t+h}, \bfo_{t+h}, R_{t+h}, \varnothing) \}\}_{j=1}^{n}.
\end{equation}
% 
(3) Subsequently, a new elite-set of actions \(\cE_t^{(i)}\) of size \(K\) is determined; it consists of the best \(K\) action sequence samples that maximize an aggregation of the returns from their expected states according to a reward aggregation function \(g\);
\begin{equation}
    \label{eq:icem-filter}
    \cE_t^{(i)} = \underset{\text{Best \(K\) action sequences in the augmented set of samples \(\cA^+_t\)}}{\{ \bma^{(h)}_k \in \cA^+_t :\ \mid \bma^{(h)}_j \in \cA^+_t : g(\bmtau^{(h)}_k) \leq g(\bmtau^{(h)}_j) \mid\ \leq K \}}.
\end{equation}
More information about the reward aggregation strategies used in our experiments is provided in \autoref{sec:reward-aggregation}.

(4) These new elite action sequences are finally used to refine the distribution to maximize the expected return of its samples.
\vspace{-1.5pt}
\begin{align}
    \label{eq:icem-update}
    \bmmu_{t}^{(i)} &= \eta\ \bmmu_t^{(i - 1)} + (1-\eta)\bmmu^{\cE_t^{(i)}},\\
    (\bmsigma_{t}^{(i)})^2 &= \eta(\bmsigma_t^{(i - 1)})^2 + (1-\eta)(\bmsigma^{\cE_t^{(i)}})^2,
\end{align}
where \(\eta \in [0, 1]\) is a \emph{momentum} factor.

The above steps are repeated for \(m\) inner iterations at every timestep.
At the end of this process, the distribution concentrates on action sequences with high returns, and the best action sequence converges to an optimum;
\begin{equation}
    \label{eq:planning-goal}
    \bma^{(h)*} = \argmax_{\bma^{(h)}_k \in \cA^+} g(\bmtau^{(h)}_k).
\end{equation}
(5) Finally, the first action of the best elite sequence of actions \(\bma^{(h)*}_t\) is executed in the environment.
The process is repeated for a specified number of rollout steps or until a terminal state or reward/cost threshold is reached.

Since iCEM is a zero-order trajectory optimizer with a limited sample budget and finite-horizon planning, we do not necessarily converge to the global optima.
Although this does not solve the full reinforcement learning problem (infinite horizon and stochastic environments), it is very powerful in optimizing for tasks ad-lib without further adaptation, which makes it suitable for optimizing changing exploration targets, as in our case (because of noisy CLIP rewards; discussed later in \secref{sec:inference-noise}).

The colored noise of the iCEM distribution leads to temporally correlated action sequences over the horizon for smoother trajectories, which helps mitigate the problems due to the sparse nature of our semantics entropy reward.
Furthermore, the addition of memory gives it good sample efficiency, which was essential for our case due to the computational bottleneck because of the long inference times for CLIP.
These reasons made iCEM a suitable choice for our experiments.

For this to be successful, a good transition model \(f\) needs to be learned, and the reward function needs to be known or discovered.
In our case, the reward function is well-defined, and to evaluate the efficacy of our entropy minimization objective in achieving good semantic expression,
we do planning using ground truth (GT) models, i.e. with access to the true simulator of the environment itself for planning.
Thus, we can perform multi-horizon planning without accumulating any model errors, which allows us to better investigate the global/local optima of our semantics reward. 

\subsection{Reward Aggregation Strategies}
\label{sec:reward-aggregation}
There are many different ways to aggregate the rewards of the trajectory samples over the horizon.
This is treated as one of the hyperparameters in our experiments.
\vspace{-12.5pt}
\begin{align}
\intertext{A trivial reward aggregation function is the sum of the rewards over the horizon (\texttt{sum}),}
\texttt{sum}(\bmtau^{(h)}) &= \sum_{k=1}^{h} R_k. \label{eq:reward-aggregation-sum}
\intertext{We further experimented with a discount factor \(\gamma \in \nR^+\) over the horizon, but usually set it to \(1.0\);}
\texttt{sum-\emph{\(\gamma\)}}(\bmtau^{(h)}) &= \sum_{k=1}^{h} \gamma^{k-1} R_k. \label{eq:reward-aggregation-sum-discount}
\intertext{However, this type of consolidating aggregation strategy is not suitable for cases where an increase in return can be preceded by an initial decrease (as a consequence of leaving local optima). So we also considered the \texttt{best} method, which uses the best return over the planning horizon;}
\texttt{best}(\bmtau^{(h)}) &= \max_{k=1}^{h} R_k. \label{eq:reward-aggregation-best}
\intertext{Additionally, to combine the better qualities of both the \texttt{sum} and \texttt{best} methods; we formulated a dynamic reward aggregation method that ranks trajectories based on the highest sum of consecutive rewards in a window of a predefined hyperparamterized length \(l\), \texttt{best-\emph{l}};}
\texttt{best-\emph{l}}(\bmtau^{(h)}) &= \max_{k=1}^{h-l+1} \sum_{j=k}^{k+l-1} R_{k+j}. \label{eq:reward-aggregation-best-dynamic}
\intertext{Another strategy is to assess the trajectories based on their last states (\texttt{last}), i.e. to simply consider only what the agent could potentially achieve at the end of the horizon.
This is particularly useful when inferring the reward is expensive, as in our case, because only one final state needs to be evaluated;}
\texttt{last}(\bmtau^{(h)}) &= R_h. \label{eq:reward-aggregation-last}
\intertext{Finally, we also experimented with a combination of the \texttt{sum} and \texttt{last} methods with \texttt{last-\emph{l}}, which is the sum of the returns of the last \(l\) steps of the trajectory;}
\texttt{last-\emph{l}}(\bmtau^{(h)}) &= \sum_{k=h-l+1}^{h} R_k. \label{eq:reward-aggregation-sum-last}
\end{align}
