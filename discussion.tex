% The Discussion should begin with a statement of the important findings of the study. Subsequent sections can address technical issues, analysis of the results, and the implications of the work. Again, it is often helpful to break down the Discussion into subsections that focus on particular topics. It is proper to include a section that summarizes and expands upon conclusions that may be drawn from the work (raises open questions, proposes future studies).
\chapter{Discussion}
\label{sec:discussion}

% Summary of CLIP reward analysis

% Limitations
% - CLIP is noisy
% - Should/Could improve with better models
We did not get major performance improvements with additional prompt engineering and hyperparameter tuning.
We think that these failures are related to capability limitations in the CLIP model we use.
This is similar to the findings observed by \cite{vlmrm} in their use of CLIP as a source of goal-conditioned rewards.

\todo{Summary of things. Things that could be helpful.}

% Comments on regularization

% RaIR was very helpful
We observed that the complementary RaIR reward was very helpful in improving the quality of the semantics reward.
It helped with grounding and coalescing the structures in more human-recognizable forms and patterns.

% Comments on the environments

% Future work
Perhaps using bigger models from OpenCLIP such as \texttt{ViT-bigG-14} \citep{openclip} which are trained on the LAION-5B dataset \citep{laion5b} might lead to some performance improvements.

% Comments on creativity

% Comments on exploration
When used as an additional reward to augment traditional novelty-seeking intrinsic rewards \(r_{\text{intrinsic}} = r_{\text{semantics}} + r_{\text{novelty}}\), it can lead to more human-like exploration i.e. curious behaviors with semantically-sound creative expression.
