% The Abstract should be concise, consisting of 200 words or less. It should briefly frame the biological problem that the paper addresses indicate in brief the method of approach, and the species of animal used, and provide a concise summary of the major results and conclusions (no subheadings, no references to the literature).
\begin{abstract}

\noindent
Humans have a propensity to explore their environments in expressive manners that manifest their previous knowledge and understanding of the world.
Particularly, we tend to exhibit a semantics bias when there is no definite goal at hand, i.e. during free play. For example, when playing with building blocks or doodling, we tend to create patterns and structures that are objectively meaningful such as a house, a car, or a tree.
In this study, we tried to realize this semantically expressive behavior in artificial agents. We formulated an entropy-based intrinsic reward for self-supervised exploration during free play using large vision language models (VLMs).
This was used in a model-based planning paradigm in special environments with building blocks as their basic elements, which enabled rich creative expression.
With the right parameters, an agent optimizing its actions for this reward was able to create a diverse set of abstract creative structures reminiscent of real objects.
To improve our results, we further experimented with different regularization methods and investigated if a complementary reward for regularity and compression rooted in a similar bias towards symmetry promoted semantic expression. 
This work provides a novel perspective on imbuing creativity in artificial agents and furthers the use of VLMs as a source of semantic rewards in robotics and AI.

% \\~\\\noindent\emph{* % additional note}
\end{abstract}