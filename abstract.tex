% The Abstract should be concise, consisting of 200 words or less. It should briefly frame the biological problem that the paper addresses indicate in brief the method of approach, and the species of animal used, and provide a concise summary of the major results and conclusions (no subheadings, no references to the literature).
\begin{abstract}

\noindent
Humans have a propensity to explore their environments in expressive manners that exhibit their knowledge and understanding of the world.
Particularly, we tend to exhibit this semantics bias when there is no definite goal at hand, i.e. during free play.
In this study, we tried to realize semantically expressive behavior in artificial agents by formulating an intrinsic reward for self-supervised exploration during free play using large vision language models (VLMs).
This reward was used in a model-based planning setup in specific environments with building blocks as their basic elements, which enabled rich creative expression.
With the right parameters, the agent was able to converge to a diverse set of abstract creative structures reminiscent of real objects.
We further investigated if an additional reward for regularity and compression promoted this semantic expression. 
This work provides a novel perspective on imbuing creativity in artificial agents and furthers the use of VLMs as a source of rewards in robotics and AI.

% \\~\\\noindent\textit{* % additional note}
\end{abstract}
