% The Abstract should be concise, consisting of 200 words or less. It should briefly frame the biological problem that the paper addresses indicate in brief the method of approach, and the species of animal used, and provide a concise summary of the major results and conclusions (no subheadings, no references to the literature).
\begin{abstract}

\noindent
Humans have a propensity to freely explore their environments in expressive manners that manifest their previous knowledge and understanding of the world.
Particularly, we tend to exhibit this semantics bias when there is no definite goal at hand, i.e. during free play.
For example, when playing with building blocks or doodling, we prefer to create patterns and structures that are objectively meaningful to us, such as depictions of a house, a car, or a tree.
In this study, we try to realize this emergent and semantically expressive behavior in artificial agents.
To this end, we formulate an entropy-based intrinsic reward for self-supervised exploration during free play using large vision language models (VLMs).
We use it in a model-based planning paradigm in two rich environments with building blocks as basic elements where we demonstrate that an agent optimizing its actions for these rewards can consistently create a range of diverse creative arrangements that are semantically expressive and reminiscent of real objects.
To improve the reward quality, we experiment with different regularization methods and investigate if a complementary reward for regularity rooted in a similar bias towards symmetry and compression promotes semantic expression.
This work improves upon and furthers the methods of using VLMs as a source of rewards and presents a novel perspective on imbuing emergent and creative behaviors in artificial intelligence.

% \\~\\\noindent\emph{* % additional note}
\end{abstract}